\section{Load, Performance and Stress Test}

Load, performance and stress testing is typically done to locate bottlenecks in a system, to support a performance tuning effort and to collect other performance-related indicators to help stakeholders get informed about the quality of the application being tested \cite{Sandler2004} \cite{Corporation2007}. 

The Performance Test aims at verifying a specified system performance. This kind of test is executed by simulating hundreds or more simultaneous users  over a defined time interval \cite{DiLucca2006}. The purpose of this test is to demonstrate that the system  reaches its performance objectives \cite{Sandler2004}. 


In load tests, the system is evaluated in pre-defined load levels \cite{DiLucca2006}. The aim of this test is to reach the performance targets for availability, concurrency, throughput and response time of the system. Load Test is the closest to real application use \cite{Molyneaux2009}.

Stress test verifies the system behaviour against heavy workloads \cite{Sandler2004}, being executed to evaluate a system beyond its limits, validate system response in activity peaks and verify if the system is able from recover from these conditions. They differ from other kinds of testing  because the system is executed on or beyond its breakpoints, forcing the application or the supporting infrastructure to fail \cite{DiLucca2006} \cite{Molyneaux2009}.

Automated tools are needed to carry out serious load, stress and performance testing. Sometimes , there is simply no practical way to provide reliable, repeatable performance tests without using some form of automation. The aim of any automated test tool is to simplify the testing process \cite{Molyneaux2009}.

In the context of  testing, a scenario is a sequence of steps in your application. It can represent a use case or a business function such as searching a product catalog, adding an item to a shopping cart or placing an order \cite{Corporation2007}. 

Load, Performance and Stress results are measured by indicators. Some researchers advocate the 90-percentile response time is a better measurement than the average/medium response time, as the 90-percentile accounts for most of the peaks, while eliminating the outliers \cite{Jiang2010}.